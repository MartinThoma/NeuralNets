\section{Distortion Measures}\label{sec:lvq-distortion-measures}
Literature: \cite{Linde1980}

Distortion is an nonnegative measure for an input vector $x$ and a reproduction vector $y$. Distortion measures may not be true metrics, \eg be unsymmetric or not fulfill the triangle inequality\footnote{triangle inequality: $d(x, y) \leq d(x, y) + d(y, y)$, for all $y$}.

Most common is the squared-error distortion, better known as squared Euclidean distance:
\begin{equation}
d(x, y) = ||x-y||_2^2 = (x-y)^T (x-y) = \sum_{i=1}^N (x_i-y_i)^2
\end{equation}

Other common distortion measures are the $l_\nu$, or Holder norm,
\begin{equation}\label{eq:holder-norm}
d(x, y) = \left( \sum_{i=1}^N |x_i - y_i|^\nu \right) ^{\frac{1}{\nu}} = || x - y ||_\nu
\end{equation}

and its $\nu^{th}$ power, the $\nu^{th}$-law distortion:
\begin{equation}
d(x, y) = \sum_{i=1}^N |x_i - y_i|^\nu
\end{equation}

The holder Norm (\ref{eq:holder-norm}) is a distance and fulfills the triangle inequality, the $\nu^{th}$-law distortion not.

All three and many others, as the weighted-squares distortion and the quadratic distortion, only depend on the difference $(x - y)$ and can be described with $d(x, y) = L(x - y)$. A distortion not having this form is the one by Itakura, Saito and Chaffee,

\begin{equation}
d(x, y) = (x - y)\, R(x)\, (x - y)^T
\end{equation}

, where $R(x)$ is the autocorrelation matrix.

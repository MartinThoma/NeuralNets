\section{k-Means (Lloyd's Algorithm)}
Finding $C^*(i)$ by enumeration is too time-consuming. Instead we will use iterative greedy descent which leads to a local optimum.
The error function is defined as
\begin{equation}
E = \sum_{k=1}^K \sum_{C(i)=k} d(x_i, \mu_k)
\end{equation}

Using the squared Euclidean distance, $d(x_i, \mu_k) = ||x_i - \mu_k||^2$, will effectively assign each sample to the closet center.

A slightly different definition is to minimize our already defined $W(C)$
\begin{equation}
W(C) = \frac{1}{2} \sum\limits_{k=1}^K \sum\limits_{C(i)=k} \sum\limits_{C(j)=k} ||x_i - x_j||^2 = \sum\limits_{k=1}^K N_k \sum\limits_{C(i)=k} ||x_i - \mu_k||^2
\end{equation}
\todo{Add derivate for this equation and why $N_k$ is left out later}

Also called \emph{Lloyd's Algorithm} we optimize for $J$ in a greedy fashion.
\begin{enumerate}
\item \textbf{Initialize}: There are many ways to initialize the centroids:
	\begin{itemize}
	\item distribute uniformly
	\item randomly (different distribution are possible)
	\item start with one random centroid and iteratively place to next one by maximizing the distance to all others
	\end{itemize}
\item \textbf{Classify}: Assign each observation $i$ to the nearest centroid: $$C(i) = \argmin\limits_{1\leq k \leq K} ||x_i - \mu_k||^2$$
\item \textbf{Recenter}: For each class $k$, compute a new centroid as the mean of the updated class assignments: $$\mu_k = \frac{\sum\limits_{C(i)=k} x_i}{N_k}$$
\item \textbf{Repeat step 2 and 3 until stopping criteria fulfilled} (\eg centers stop moving, or equally until C(i) stays unchanged).
\end{enumerate}

During the course of the k-means algorithm, the error function monotonically decreases into a local minimum. The proof is simply done by proofing that both steps will independently decrease $E$ if possible.

\section{Fuzzy k-Means}\label{sec:fuzzy-kmeans}
Literature: \cite{Introduction2000}\\
In English literature it is sometimes also called \emph{Fuzzy c-Means} (FCM).

Observation does not belong strictly to one cluster, but has each $x_i$ a member ship degree $u_{ik}$ for each cluster~$k$.
\begin{align}
0 \leq u_{ik} \leq 1 \quad \forall i, k\\
\sum_k u_{ik} = 1 \quad \forall i\\
\end{align}

Our new error function is
\begin{equation}\label{eq:fuzzy-kmeans-error}
E(m) = \sum_{i=1}^N \sum_{k=1}^K u_{ik}^m d(x_i, \mu_k)
\end{equation}
with fuzzifier $m \geq 1$ that controls fuzziness of the clusters. With $m=1$ the membership degrees will converge to $0$ or $1$, which implies a crisp partitioning. In the absence of experimentation or domain knowledge, $m$ is commonly set to $2$.

\begin{enumerate}
\item \textbf{Compute degree of membership:}
	$$u_{ik} = \left[ \sum_{j=1}^K \left( \frac{d(x_i, \mu_k)}{d(x_i, \mu_j)} \right)^{\frac{2}{m-1}} \right]^{-1}$$
\item \textbf{Recenter:}
	$$\mu_k = \frac{\sum_{i=1}^N u_{ik}^m \, x_i}{\sum_{i=1}^N u_{ik}^m}$$
\item \textbf{Repeat step 1 and 2 until stopping criteria fulfilled.}
\end{enumerate}

\section{LBG-Algorithm}
Literature: \cite{Linde1980}, \href{http://www.data-compression.com/vqanim.shtml}{Online Demo}

The Linde, Buzo, Grey (LBG) Algorithm is an alternative method to design a $K$-vector codebook (codebook vector = centroid), where $K=2^x$.\\
The algorithm takes a parameter a small $\epsilon > 0$ (\eg $\epsilon=0.001$).

\begin{enumerate}
\item \textbf{Initialize} 1-vector starting codebook:
	\begin{align}
	\begin{split}
	\mu_1(0) &= \frac{1}{N} \sum\limits_{i=1}^N x_i\\
	t &= 0
	\end{split}
	\end{align}
\item \textbf{Double codebook}:
	\begin{align}
	\begin{split}
	\mu_j(t+1) &= (1+\epsilon)\, \mu_j(t) \quad \text{for } j=1,\cdots,2^t\\
	\mu_{2*j+1}(t+1) &= (1-\epsilon)\, \mu_j(t) \quad \text{for } j=1,\cdots,2^t\\
	t &= t+1
	\end{split}
	\end{align}
\item \textbf{Run k-Means} Use k-Means to optimize to improve the $\mu_j(t)$.
\item \textbf{Repeat step 2 and 3} until desired number of codebook vectors is reached (for $t=1,\cdots,\log_2(K)-1$)
\end{enumerate}

Our final codebook will be $\mu_j(\log_2(K))$.

This simple form of the \emph{LBG-Algorithm} doesn't have any advantages over k-means, but if splitting of the centroids is done smart, the final centroids are more likely to be a good representation of the data:
\begin{itemize}
\item k-Means initialization is random and not data dependent. Small groups of samples might get equal amount or even more centroids than big groups.
\item $\epsilon$ can be set do split the centroids according to the direction with the highest variance for each dimension.
\item Instead of splitting all centroids only the  with biggest variance or biggest number of assigned samples can be splitted.
\end{itemize}

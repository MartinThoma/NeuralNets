\subsection{k-Means}
% Finding $C^*(i)$ by enumeration is too time-consuming.
% ! Finding C*(i) by enumeration is too time-consuming
% tance can be used by􏰀redefining the x values (Exercise 14.1). 2 ij2
% d(x,x′)= (x −x′ ) =||x −x′|| ! Instead use itei raitive greeidj y diejscent i i
% p
% The K-means algorithm is one of the most popular iterative des The within-point scatter (14.28) can be written as
% j=1 ! Convergenctertiongamloceathlodptsi.mIatisintendedforsituationsinwhichallvariab
% ! Dissimilarity measure
Minimize $$W(C) = \frac{1}{2} \sum\limits_{k=1}^K \sum\limits_{C(i)=k} \sum\limits_{C(j)=k} ||x_i - x_j||^2 = \sum\limits_{k=1}^K N_k \sum\limits_{C(i)=k} ||x_i - \mu_k||^2$$

\subsection{Lloyd's Algorithm}
\begin{enumerate}
\item \textbf{Classify}: Assign each observation $i$ to the nearest centroid: $$C(i) = \argmin\limits_{1\leq k \leq K} ||x_i - \mu_k||^2$$
\item \textbf{Recenter}: For each class $k$, compute a new centroid as the mean of the updated class assignments: $$\mu_k = \frac{\sum\limits_{C(i)=k} x_i}{N_k}$$
\item \textbf{Repeat until stopping criteria fulfilled}
\end{enumerate}
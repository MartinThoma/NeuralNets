\section{Notation}\label{sec:notation}
Throughout the literature (and history) in the big field of neural networks you will find a whole bunch of different notations and meanings of variable names. Especially when the meaning changes inside a document it can be very confusing for beginners. For this document I will try to stick with one notation. I found it to be very flexible and consistent.

\begin{tabularx}{\textwidth}{ l X }
Symbol			& Meaning(s) \\ \midrule
% parameters
$m$				& Number of features, size of input vector \\
$k$, $K$		& Number of target classes/clusters, size of output vector \\
$n$, $N$		& Number of training examples/observations, size of training set \\
$\eta$			& Learning rate (see \ref{sec:learning-rate}) \\
$\theta$		& Represents all model parameters \\
$X$				& Training data \\

\midrule % VQ
$\mu_j$			& Centroid of cluster~$j$, also called codebook vector (with $\mu$ as codebook) \\
$d(x, y)$		& Dissimilarity measure, distance or distortion measure, if mention otherwise we use the squared Euclidean distance (see \ref{sec:lvq-distortion-measures}) \\

$E$, $J$		& Error function (see \ref{sec:error-functions}), other names are: loss function, objective function or criterion function\newline
					By default we will use the mean squared error. \\

\midrule % MLP
$w_{ij}^{(l)}$	& Weight of connection from neuron~$i$ in layer~$(l-1)$ to neuron~$j$ in layer~$l$, if the layer is omitted all variables in the equation are from the same layer \\
$\mathbf{W}$, $\mathbf{w}$ & Weight matrix, weight vector \\
$b_j^{(l)}$	& bias unit for neuron~$j$ in layer~$l$ \newline For convenience we will use different meanings throughout the lecture notes:
				\begin{itemize}
				\item Often the bias is not important for the understanding and left out, but keep in mind that neural network implementation nearly always require bias for good results.
				\item $b$ has an associated weight (usually $w_{0j}$), than the value of $b$ is always $1$ and we only train the weight
				\item if there is no weight, we adjust $b_j$ during training
				\end{itemize} \\
$\Sigma$		& Summation operator (\eg $\sum_{i=1}^n x_i$), covariance matrix or the input function of a neuron \\
$\phi(x)$		& Activation function (see \ref{sec:activation-functions}) \\
$\sigma(x)$		& Sigmoid activation function (see \ref{sec:sigmoid}) \\
$z_j^{(l)}$		& Total input of the neuron~$j$ in layer~$l$, usually $z_j^{(l)} = b_j^{(l)} + \sum_i w_{ij}^{(l)} a_i^{(l-1)}$ \\
$a_j^{(l)}$		& Activation/Output of neuron~$j$ in layer~$l$, usually computed from the total input $z_j^{(l)}$, $a_j=\phi(z_j)$ \\

\midrule % BM
$v_j \in \{0,1\}$ & Activation of visible neuron/unit~$j$ in Boltzmann machine \\
$h_j \in \{0,1\}$ & Activation of hidden neuron/unit~$j$ in Boltzmann machine \\
$b_j (c_j)$		& Bias of visible (hidden) neuron~$j$ in \gls{RBM} \\

\midrule % Other
$\delta_{ij}$	&  Kronecker delta, which is defined as $\delta_{ij}=\begin{cases}1 &\text{if } i = j,\\0 &\text{if } i \neq j.\end{cases}$
\end{tabularx}